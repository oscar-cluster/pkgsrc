%%%%%%%%%%%%%%%%%%%%%%%%%%%%%%%%%%%%%%%%%%%%%%%%%%%%%%%%%%%%%%%%%%%%%%%%%%%%%%%
%%  Documentation for kernel_picker.  You must process this file with        %%
%%  latex2man for proper output.  latex2man is available at:                 %%
%%       http://www.ctan.org/tex-archive/support/latex2man/                  %%
%%  For RedHat Linux, download the .tar.gz file, extract the files, and      %%
%%  copy all of the latex2man.* files to the following directory:            %%
%%       /usr/share/texmf/tex/latex/latex2man                                %%
%%  Then type "texhash".  You should now be able to run latex on this file   %%
%%  and do whatever you want.  Some possible outputs:                        %%
%%  * man page   : latex2man kernel_picker.tex kernel_picker.1               %%
%%  * HTML       : latex2man -H kernel_picker.tex kernel_picker.html         %%
%%  * PostScript : latex kernel_picker.tex; dvips kernel_picker.dvi          %%
%%  * PDF        : pdflatex kernel_picker.tex                                %%
%%  * Plain text : Use lynx's "p(rint)" command to "Save to a local file" OR %%
%%                 man ./kernel_picker.1 > kernel_picker.txt                 %%
%%%%%%%%%%%%%%%%%%%%%%%%%%%%%%%%%%%%%%%%%%%%%%%%%%%%%%%%%%%%%%%%%%%%%%%%%%%%%%%
%%  $COPYRIGHT$

\documentclass[english]{article}
\usepackage[latin1]{inputenc}
\usepackage{babel}

%% do we have the `fancyhdr' package?
\IfFileExists{fancyhdr.sty}{
\usepackage[fancyhdr,pdf]{latex2man}
}{
%% do we have the `fancyheadings' package?
\IfFileExists{fancyheadings.sty}{
\usepackage[fancy,pdf]{latex2man}
}{
\usepackage[nofancy,pdf]{latex2man}
\message{no fancyhdr or fancyheadings package present, discard it}
}}

%% do we have the `rcsinfo' package?
\IfFileExists{rcsinfo.sty}{
\usepackage[nofancy]{rcsinfo}
\rcsInfo $Id: kernel_picker.tex,v 1.2 2003/09/03 22:15:40 tfleury Exp $
\setDate{\rcsInfoLongDate}
}{
\setDate{2003/09/03}    %%%% must be manually set, if rcsinfo is not present
\message{package rcsinfo not present, discard it}
}

\setVersionWord{Version:}  %%% that's the default, no need to set it.
\setVersion{1.4}

\begin{document}

\begin{Name}{1}{kernel\_picker}{Terry Fleury}{OSCAR Tools}{kernel\_picker\\--\\ An OSCAR Helper Tool}

  \Prog{kernel\_picker} is a Perl script which allows the user to install
  a given kernel into an OSCAR image different from the one which is
  installed by default.
\end{Name}

\section{Synopsis}
%%%%%%%%%%%%%%%%%%

\Prog{kernel\_picker} 
    \oOptoArg{--oscarimage }{image\_name}
    \oOptArg{--bootkernel }{boot\_kernel}
    \oOptArg{--bootlabel }{boot\_label}
    \oOptoArg{--bootramdisk }{Y/n}
    \oOptoArg{--networkboot }{Y/n}
    \oOptArg{--kernelversion }{version\_number}
    \oOptArg{--modulespath }{modules\_path}
    \oOptArg{--systemmap }{system\_map}
    \oOpt{--version}
    \oOpt{--help}

\section{Description}
%%%%%%%%%%%%%%%%%%%%%
\Prog{kernel\_picker} allows you to substitute a given kernel into your
      OSCAR (SIS) image prior to building your nodes.  If executed with no
      command line options, you will be prompted for all required
      information.  You can also specify command line options as shown
      below.  Any necessary information that you do not give via an option
      will cause the program to prompt you for that information.

      The program assumes that the optional OSCAR image files you wish to
      use reside in a subdirectory in the
      \File{/var/lib/systemimager/images} directory.  By default, the
      original OSCAR image is in a subdirectory named \File{oscarimage}.  

\section{Options}
%%%%%%%%%%%%%%%%%
\begin{Description}\setlength{\itemsep}{0cm}
\item[\oOptoArg{--oscarimage }{image\_name}] The name of the OSCAR image
     directory.  If you use this option but do not specify
     \Arg{image\_name}, the default directory '\File{oscarimage}' will be
     used.
\item[\oOpt{--autoselectimage}] In the event that there are two or more
     OSCAR image directories but you do not want to have to specify one via
     the '--oscarimage' option (e.g. when you want to call
     \Prog{kernel\_picker} from another program), use this option.  (The
     first OSCAR image directory, alphabetcially ignoring case, will be
     used.) Otherwise, you will be presented with a list of OSCAR image
     directories and be asked to choose one to use.
\item[\oOptArg{--bootkernel }{boot\_kernel}] The full path specifier of the
     kernel to use at boot time.  This kernel file will be copied to the
     system's boot directory.  Also, \File{systemconfig.conf}'s PATH option
     will be set approriately.
\item[\oOptArg{--bootlabel }{boot\_label}] The label for the kernel to use at
     boot time.  This value is set as the LABEL option in the
     \File{systemconfig.conf} file.  If you provide the \Arg{boot\_kernel}
     but do not provide the \Arg{kernel\_label}, the \Arg{kernel\_label}
     defaults to the file name specified by the \Arg{boot\_kernel} (without
     any leading path information).
\item[\oOptoArg{--bootramdisk }{Y/n}] Whether or not to configure a ram disk
     for booting.  This value is set as the CONFIGRD option in the
     \File{systemconfig.conf} file.  If you do not specify Y or N, it
     defaults to YES.
\item[\oOptoArg{--networkboot }{Y/n}] Whether or not to use the specified
     \Arg{boot\_kernel} for network booting the nodes during the build
     process.  If so, the kernel gets copied to '\File{/tftpboot}'.  If you
     do not specify Y or N, it defaults to YES.
\item[\oOptArg{--kernelversion }{version\_number}] If your boot kernel uses
     loadable modules, you must provide the full version number/name of the
     kernel.  This is the value output by the Unix command 
     '\Prog{uname -r}'.
\item[\oOptArg{--modulespath }{modules\_path}] If your boot kernel uses
     loadable modules, you must profide the source directory containing the
     \File{/lib/modules/version\_number} directory tree.  There is no
     need to prepend \File{/lib/modules/version\_number} to this
     option (but it doesn't hurt if you do).
\item[\oOptArg{--systemmap }{system\_map}] If your boot kernel uses loadable
     modules, you may optionally provide the full path specifier of the
     \File{System.map} file to use by depmod.  Enter the full path/file name
     of the source file.  This file will be copied to the boot directory.
\item[\oOpt{--version}] Prints out the version string of the program and
     then quits.
\item[\oOpt{--help}] Prints out a help message and then quits.
\end{Description}


\section{General Remarks}
%%%%%%%%%%%%%%%%%%%%%%%%%%%%
\begin{enumerate}\setlength{\itemsep}{0cm}
\item \Prog{kernel\_picker} does not modify any configuration on the host
      machine.
\item When you run the program without any command line options, you will be
      prompted for all required information.  If you use any command line
      options, the program will assume that you know what you are doing and
      prompt you ONLY for information which is required for correct
      execution.  For example, if you specify the \Opt{--bootlabel} but not
      any of the options required for loadable modules
      (\Opt{--kernelversion}, \Opt{--modulespath}, and \Opt{--systemmap}),
      the program will assume that your kernel does not require loadable
      modules and will not prompt you for any of those options.  
\item All of the required information is gathered prior to acutally running
      any commands.  So you can abort execution before the end and no files
      will be modified/copied.
\item Many of the questions have a default value indicated by bracketed text
      at the end of the question.  If this value is okay, simply hit
      $<$ENTER$>$ to accept it.  For most of the questions, the default
      value is probably what you want.
\item If there is only one subdirectory in the
      \File{/var/lib/systemimager/images} directory, it will be used as the
      \Opt{--oscarimage}.  
\item When you enter the \Arg{boot\_kernel}, it will be scanned for a
      version number string.  If found, this will be used as the default
      values for the \Arg{kernel\_label} and \Arg{version\_number}.  You can
      override this by entering your own label and version.
\item If you want to run the program from the command line and not be
      prompted for any information, you should invoke the program as
      follows:

      \Prog{kernel\_picker} 
      \OptArg{--bootkernel}{ boot\_kernel} 
      \Opt{--bootramdisk} N 
      \Opt{--networkboot} N
\end{enumerate}

\section{Requirements}
%%%%%%%%%%%%%%%%%%%%%%
\begin{description}\setlength{\itemsep}{0cm}
\item[Perl] The \Prog{kernel\_picker} script requires Perl version 
     $>=$ 5.5.
\item[Initrd::Generic.pm] In order to extract the kernel version number
     string from the \Arg{boot\_kernel}, \Prog{kernel\_picker} requires the
     \File{Initrd} package.  This is usually found in the
     \File{/usr/lib/systemconfig} directory and is part of the
     \File{systemconfigurator} RPM package.
\end{description}

\section{Version}
%%%%%%%%%%%%%%%%%
Version: \Version\ of \Date.

\section{Author}
%%%%%%%%%%%%%%%%
\noindent
Terry Fleury                      \\
Email: \Email{tfleury@ncsa.uiuc.edu} 
\LatexManEnd

\end{document}
